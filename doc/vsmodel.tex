\documentclass[]{article}
\usepackage{natbib}
\usepackage{amsmath}
\usepackage{amsfonts}
\usepackage{amssymb}
%opening
\title{\texttt{vsmodel} Derivation}
\author{Matt James}

\begin{document}

\maketitle

This document describes the derivation of the model equations for the \texttt{vsmodel} package.

\section{Derivation of the Model}

In this section we derive the components of the Volland-Stern electric field model \citep{Volland1973,Stern1975} in cylindrical coordinates. The Volland-Stern model starts with an electrostatic potential, $U(r,\phi)$, from which the electric field is obtained using

\begin{equation}
	\mathbf{E} = - \mathbf{\nabla} U, \label{EqEField}
\end{equation} 

where

\begin{equation}
	\mathbf{\nabla} U = \frac{\partial U}{\partial r} \mathbf{\hat{r}} + \frac{1}{r} \frac{\partial U}{\partial \phi} \mathbf{\hat{\phi}} + \frac{\partial U}{\partial z}\mathbf{\hat{z}}. \label{EqDelU}
\end{equation}

The coordinate system is such that $z$ lies along the dipole axis of the planet; $x$ points approximately sunward where the magnetic equatorial plane intersects the plane containing both the dipole axis and the Earth-Sun line; the $y$ axis points approximately duskward. In cylindrical coordinates: $r$ is the radial distance from the $z$ axis (i.e. $r = \sqrt{x^2 + y^2}$); $\phi$ is the azimuth, equal to 0 at noon where $\phi = \arctan{y,x}$.

The potential, $U$, is made up of a corotational component, $U_{cor}$, and a convection component, $U_{cnv}$. The corotation component used here is defined by

\begin{equation}
	U_{cor} = -\frac{a}{r},
\end{equation}

where $a=92.4$~keV is the corotation constant used in \citet{Zhao2017}.

The electric field due to corotation is given by:
\begin{align}
	E_r &= -\frac{a}{r^2}, \\
	E_\phi &= 0,\\
	E_z &= 0.
\end{align}

There are two options for the convection component of this model - the simpler option is from \citet{Maynard1975} and is derived in section \ref{SectMC}; the more complicated option described by \citet{Goldstein2005} separates the convection component into solar wind electric field (section \ref{SectSW}) and SAPS (section \ref{SectSAPS}) parts.


\section{The \citet{Maynard1975} $\mathbf{E}$-field}

This section describes the electric field due to convection as described by \citet{Maynard1975}. The potential used is

\begin{equation}
	U = -A_{mc} r^\gamma \sin{\phi}, \label{EqMCPot}
\end{equation}

where 

\begin{equation}
	A_{mc} = \frac{0.045}{(1.0 - 0.159 Kp + 0.0093 K_p^2)^3} \text{ (kV $R_E^{-2}$)}
\end{equation}

and $\gamma=2$ is the shielding parameter.

The electric field components become
\begin{align}
	E_r &= \gamma A_{mc} r^{(\gamma-1)}\sin{\phi},\\
	E_\phi &= A_{mc} r^{(\gamma-1)} \cos{\phi},\\
	E_z &= 0.
\end{align}

\section{Solar Wind Electric Field}

This section uses the electric field due to the solar wind propagation past the Earth as described in \citet{Goldstein2005}. The electric field due to the solar wind is given by,

\begin{equation}
	E_{sw} = -V_{sw} B_z,
\end{equation}

where $V_{sw}$ is the $x$ component of the solar wind velocity (negative Sunward), and $B_z$ is the north-south component of the interplanetary magnetic field (IMF). $E_{sw}$ has a minimum value of 0.1~mV~m$^{-1}$, so when the IMF is northward there is still a little bit of a viscous interaction with the magnetosphere.

The potential is given by 
\begin{equation}
	U = -A_{sw} r^2 \sin{\phi},
\end{equation}
where 
\begin{equation}
	A_{sw} = 0.12 E_{sw} (6.6)^{(1-\gamma)}, \text{ (kV $R_E^{-2}$)}
\end{equation}
and $E_{sw}$ in this case should be converted from mV~m$^{-1}$ to kV~$R_E^{-1}$.

The electric field components become
\begin{align}
E_r &= \gamma A_{sw} r^{(\gamma-1)}\sin{\phi},\\
E_\phi &= A_{sw} r^{(\gamma-1)} \cos{\phi},\\
E_z &= 0.
\end{align}


\section{SM Model Field}

	The previous section describes the model in polar/cylindrical coordinates (they are equivalent at $z=0$), here we convert the model to SM coordinates by rotating about the $z$-axis.
	
	Considering an Electric field vector $\mathbf{E}(r,\phi)$ with components $E_r$ and $E_\phi$ -- we need to rotate this vector by $\phi$ to transform into SM coordinates.
	
	Start by expressing the components of $\mathbf{E}$ in terms of some polar coordinates $\rho$ and $\alpha$:
	
	\begin{align}
		E_r &= \rho \cos{\alpha}, \label{EqEr}\\
		E_\phi &= \rho \sin{\alpha}, \label{EqEp}
	\end{align}
	
	then rotate by $\phi$,
	
	\begin{align}
		E_x &= \rho \cos{(\alpha + \phi)}, \label{EqEx0}\\
		E_y &= \rho \sin{(\alpha + \phi)}. \label{EqEy0}		
	\end{align}
	
	Using the trigonometric identities,
	
	\begin{align}
		\sin{(\alpha \pm \phi)} &= \sin\alpha\cos\phi \pm \cos\alpha\sin\phi, \\
		\cos{(\alpha \pm \phi)} &= \cos\alpha\cos\phi \mp \sin\alpha\sin\phi,
	\end{align}
	
	equations \ref{EqEx0} and \ref{EqEy0} become 
	
	\begin{align}
		E_x &= \rho\cos\alpha\cos\phi - \rho\sin\alpha\sin\phi, \\
		E_y &= \rho\sin\alpha\cos\phi + \rho\cos\alpha\sin\phi.
	\end{align}
	
	Then, substituting in equations \ref{EqEr} and \ref{EqEp}, gives
	
	\begin{align}
		E_x &= E_r\cos\phi - E_\phi\sin\phi, \\
		E_y &= E_r\sin\phi + E_\phi\cos\phi.
	\end{align}

\bibliographystyle{agu08}
\bibliography{references}
\end{document}
