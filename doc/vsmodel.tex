\documentclass[]{article}
\usepackage{natbib}
\usepackage{amsmath}
\usepackage{amsfonts}
\usepackage{amssymb}
%opening
\title{\texttt{vsmodel} Derivation}
\author{Matt James}

\begin{document}

\maketitle

This document describes the derivation of the model equations for the \texttt{vsmodel} package.

\section{Derivation of the Model}

In this section we derive the components of the Volland-Stern electric field model \citep{Volland1973,Stern1975} in cylindrical coordinates. The Volland-Stern model starts with an electrostatic potential, $U(r,\phi)$, from which the electric field is obtained using

\begin{equation}
	\mathbf{E} = - \mathbf{\nabla} U, \label{EqEField}
\end{equation} 

where

\begin{equation}
	\mathbf{\nabla} U = \frac{\partial U}{\partial r} \mathbf{\hat{r}} + \frac{1}{r} \frac{\partial U}{\partial \phi} \mathbf{\hat{\phi}} + \frac{\partial U}{\partial z}\mathbf{\hat{z}}. \label{EqDelU}
\end{equation}

The coordinate system is such that $z$ lies along the dipole axis of the planet; $x$ points approximately sunward where the magnetic equatorial plane intersects the plane containing both the dipole axis and the Earth-Sun line; the $y$ axis points approximately duskward. In cylindrical coordinates: $r$ is the radial distance from the $z$ axis (i.e. $r = \sqrt{x^2 + y^2}$); $\phi$ is the azimuth, equal to 0 at noon where $\phi = \arctan{y,x}$.

The potential, $U$, is made up of a corotational component, $U_{cor}$, and a convection component, $U_{cnv}$. The corotation component used here is defined by

\begin{equation}
	U_{cor} = -\frac{a}{r},
\end{equation}

where $a=92.4$~keV is the corotation constant used in \citet{Zhao2017}.

The electric field due to corotation is given by:
\begin{align}
	E_r &= -\frac{a}{r^2}, \\
	E_\phi &= 0,\\
	E_z &= 0.
\end{align}

There are two options for the convection component of this model - the simpler option is from \citet{Maynard1975} and is derived in section \ref{SectMC}; the more complicated option described by \citet{Goldstein2005} separates the convection component into solar wind electric field (section \ref{SectSW}) and SAPS (section \ref{SectSAPS}) parts.


\section{The \citet{Maynard1975} $\mathbf{E}$-field}

This section describes the electric field due to convection as described by \citet{Maynard1975}. The potential used is

\begin{equation}
	U = -A_{mc} r^\gamma \sin{\phi}, \label{EqMCPot}
\end{equation}

where 

\begin{equation}
	A_{mc} = \frac{0.045}{(1.0 - 0.159 Kp + 0.0093 K_p^2)^3} \text{ (kV $R_E^{-2}$)}
\end{equation}

and $\gamma=2$ is the shielding parameter.

The electric field components become
\begin{align}
	E_r &= \gamma A_{mc} r^{(\gamma-1)}\sin{\phi},\\
	E_\phi &= A_{mc} r^{(\gamma-1)} \cos{\phi},\\
	E_z &= 0.
\end{align}

\section{Solar Wind Electric Field}

This section uses the electric field due to the solar wind propagation past the Earth as described in \citet{Goldstein2005}. The electric field due to the solar wind is given by,

\begin{equation}
	E_{sw} = -V_{sw} B_z,
\end{equation}

where $V_{sw}$ is the $x$ component of the solar wind velocity (negative Sunward), and $B_z$ is the north-south component of the interplanetary magnetic field (IMF). $E_{sw}$ has a minimum value of 0.1~mV~m$^{-1}$, so when the IMF is northward there is still a little bit of a viscous interaction with the magnetosphere.

The potential is given by 
\begin{equation}
	U = -A_{sw} r^2 \sin{\phi},
\end{equation}
where 
\begin{equation}
	A_{sw} = 0.12 E_{sw} (6.6)^{(1-\gamma)}, \text{ (kV $R_E^{-2}$)}
\end{equation}
and $E_{sw}$ in this case should be converted from mV~m$^{-1}$ to kV~$R_E^{-1}$.

The electric field components become
\begin{align}
E_r &= \gamma A_{sw} r^{(\gamma-1)}\sin{\phi},\\
E_\phi &= A_{sw} r^{(\gamma-1)} \cos{\phi},\\
E_z &= 0.
\end{align}


\section{SAPS Electric Field}

	This section uses the electrostatic potential due to SAPS as described in \citet{Goldstein2005},
	
	\begin{equation}
		U_{saps}(r,\phi,K_p) = -V_s(K_p) F(r,\phi,K_p) G(\phi),
	\end{equation}

	where
	
	\begin{align}
		V_s(Kp) &= 0.75 K_p^2 \text{ (in kV)}, \\
		F(r,\phi,K_p) &= \frac{1}{2} + \frac{1}{\pi}\arctan{\left[\frac{2}{\alpha(\phi,K_p)}\left\{ r - R_s(\phi,K_p)\right\}\right]}, \\
		G(\phi) &= \sum_{m=0}^{2} \left\{A_m \cos{[m(\phi - \phi_0)]} + B_m \sin{[m(\phi-\phi_0)]} \right\},
	\end{align}
	
	and
	
	\begin{align}
		\alpha(\phi,K_p) &= 0.15 + (2.55 - 0.27 K_p)\left[1 + \cos{\left(\phi - \frac{7\pi}{12}\right)}\right], \\
		R_s(\phi,K_p) &= R_0(K_p)\left(\frac{1+\beta}{1 + \beta\cos{(\phi-\pi)}}\right)^\kappa, \\
		R_0(K_p) &= 4.4 - 0.6(K_p - 5) \text{ (in $R_E$)},\\
		A_m &= [0.53,0.37,0.1],\\
		B_m &= [0.0,0.21,-0.1],\\
		\phi_0 &= \frac{\pi}{2}, \\
		\beta &= 0.97,\\
		\kappa &= 0.14.
	\end{align}
	
	The cylindrical components of the electric field are given by,
	
	\begin{align}
		E_r &= V_s(K_p)\frac{\partial F(r,\phi,K_p)}{\partial r} G(\phi),\\
		E_\phi &= \frac{1}{r} V_s \left[ F(r,\phi,K_p)\frac{\text{d} G(\phi)}{\text{d} \phi} + \frac{\partial F(r,\phi,K_p)}{\partial \phi} G(\phi) \right],\\
		E_z &= 0.
	\end{align}
	
	\subsection{Derivatives: $\frac{\partial F}{\partial r}$}
		\begin{align}
			\frac{\partial F}{\partial r} &= \frac{\partial}{\partial r} \left(\frac{1}{2} + \frac{1}{\pi}\arctan{\left[\frac{2}{\alpha}\left\{r - R_s\right\}\right]}\right) \\
			&= \frac{\partial}{\partial r}\left(\frac{1}{2} + \frac{1}{\pi}\arctan\left[f\right]\right) \\
			&= 0 + \frac{1}{\pi}\left(\frac{\partial}{\partial f}\arctan{f}\right)\frac{\partial f}{\partial r},
		\end{align}
		
		where
		
		\begin{equation}
			\frac{\partial }{\partial f} \arctan{f} = \frac{1}{1 + f^2},
		\end{equation}
		
		and
		
		\begin{equation}
			\frac{\partial f}{\partial r} = \frac{2}{\alpha}.
		\end{equation}
		
		So,
		
		\begin{equation}
			\boxed{\frac{\partial F}{\partial r} = \frac{2}{\alpha \pi} \left[\frac{1}{1 + f^2}\right].}
		\end{equation}
	
	\subsection{Derivatives: $\frac{\partial F}{\partial \phi}$}
		
		\begin{align}
			\frac{\partial F}{\partial \phi} &= \frac{\partial}{\partial \phi} \left(\frac{1}{2} + \frac{1}{\pi}\arctan{\left[\frac{2}{\alpha}\left\{r - R_s\right\}\right]}\right) \\
			&= \frac{\partial}{\partial \phi}\left(\frac{1}{2} + \frac{1}{\pi}\arctan\left[f\right]\right) \\
			&= 0 + \frac{1}{\pi}\left(\frac{\partial}{\partial f}\arctan{f}\right)\frac{\partial f}{\partial \phi},
		\end{align}
		
		where

		\begin{align}
			\frac{\partial }{\partial f} \arctan{f} &= \frac{1}{1 + f^2}, \\
			\frac{\partial f}{\partial \phi} &= \frac{\partial }{\partial \phi} (g(\phi,K_p) h(\phi,K_p)), \\
			&= \frac{\partial g}{\partial \phi}h + g\frac{\partial h}{\partial \phi},\\
			\frac{\partial g}{\partial \phi} &= \frac{\partial}{\partial \phi} \left(\frac{2}{\alpha}\right) = \frac{\partial g}{\partial \alpha} \frac{\partial \alpha}{\partial \phi} = -\frac{2}{\alpha^2}\frac{\partial \alpha}{\partial \phi}, \\
			\frac{\partial h}{\partial \phi} &= -\frac{\partial R_s}{\partial \phi}.
		\end{align}
		
		\begin{equation}
			\boxed{\therefore 	\frac{\partial F}{\partial \phi} = \frac{1}{\pi} \left(\frac{1}{1+f^2}\right) \left[\frac{-2}{\alpha^2}\frac{\partial \alpha}{\partial \phi} h - g \frac{\partial R_s}{\partial \phi} \right].}
		\end{equation}

		
	\subsection{Derivatives: $\frac{\partial R_s}{\partial \phi}$}
		
		\begin{align}
			\frac{\partial R_s}{\partial \phi} &= R_0 \frac{\mathrm{d}}{\mathrm{d} \phi} \left\{ \left(\frac{1+\beta}{1 + \beta\cos{(\phi-\pi)}}\right)^\kappa \right\} = R_0 \frac{\mathrm{d}}{\mathrm{d} S} \left\{ S^\kappa \right\} \frac{\mathrm{d} S}{\mathrm{d} \phi}, \\
			S &= \frac{1+\beta}{1 + \beta\cos{(\phi-\pi)}} = \frac{p}{q}, \\
			\frac{\mathrm{d}}{\mathrm{d} S} \left\{ S^\kappa \right\} &= \kappa S^{(\kappa -1)} = \kappa \left(\frac{1+\beta}{1 + \beta\cos{(\phi-\pi)}}\right)^{\kappa-1},\\
			\frac{\mathrm{d} S}{\mathrm{d} \phi} &= \frac{\frac{\text{d} p}{\text{d} \phi}q - p \frac{\mathrm{d} q}{\mathrm{d} \phi}}{q^2} = -\frac{p}{q^2} \frac{\mathrm{d} q}{\mathrm{d} \phi} = \frac{1 + \beta}{(1 + \beta\cos{(\phi - \pi)})^2} \cdot \beta \sin{(\phi - \pi)}, 
		\end{align}
		
		\begin{equation}
			\boxed{\therefore \frac{\partial R_s}{\partial \phi} = -R_0\frac{1}{q}\frac{\mathrm{d} q}{\mathrm{d} \phi} \kappa \left( \frac{p}{q} \right)^{\kappa}  = R_0 \frac{(1 + \beta)\beta \sin{(\phi - \pi)}}{(1 + \beta\cos{(\phi - \pi)})^2} \kappa \left(\frac{1+\beta}{1 + \beta\cos{(\phi-\pi)}}\right)^{\kappa-1}.}
		\end{equation}
			
		
		
	\subsection{Derivatives: $\frac{\partial \alpha}{\partial \phi}$}
	
		\begin{equation}
			\boxed{\frac{\mathrm{d} \alpha}{\mathrm{d} \phi} = -(2.55 - 0.27 K_p)\sin{\left(\phi - \frac{7\pi}{12}\right)}}
		\end{equation}
	
	\subsection{Derivatives: $\frac{\mathrm{d} G}{\mathrm{d} \phi}$}
	
		\begin{equation}
			\boxed{\frac{\mathrm{d} G}{\mathrm{d} \phi} = \sum_{m=0}^{2} \left\{-m A_m \sin{[m(\phi - \phi_0)]} + m B_m \cos{[m(\phi-\phi_0)]} \right\}.}
		\end{equation}
		
	
\section{SM Model Field}

	The previous section describes the model in polar/cylindrical coordinates (they are equivalent at $z=0$), here we convert the model to SM coordinates by rotating about the $z$-axis.
	
	Considering an Electric field vector $\mathbf{E}(r,\phi)$ with components $E_r$ and $E_\phi$ -- we need to rotate this vector by $\phi$ to transform into SM coordinates.
	
	Start by expressing the components of $\mathbf{E}$ in terms of some polar coordinates $\rho$ and $\alpha$:
	
	\begin{align}
		E_r &= \rho \cos{\alpha}, \label{EqEr}\\
		E_\phi &= \rho \sin{\alpha}, \label{EqEp}
	\end{align}
	
	then rotate by $\phi$,
	
	\begin{align}
		E_x &= \rho \cos{(\alpha + \phi)}, \label{EqEx0}\\
		E_y &= \rho \sin{(\alpha + \phi)}. \label{EqEy0}		
	\end{align}
	
	Using the trigonometric identities,
	
	\begin{align}
		\sin{(\alpha \pm \phi)} &= \sin\alpha\cos\phi \pm \cos\alpha\sin\phi, \\
		\cos{(\alpha \pm \phi)} &= \cos\alpha\cos\phi \mp \sin\alpha\sin\phi,
	\end{align}
	
	equations \ref{EqEx0} and \ref{EqEy0} become 
	
	\begin{align}
		E_x &= \rho\cos\alpha\cos\phi - \rho\sin\alpha\sin\phi, \\
		E_y &= \rho\sin\alpha\cos\phi + \rho\cos\alpha\sin\phi.
	\end{align}
	
	Then, substituting in equations \ref{EqEr} and \ref{EqEp}, gives
	
	\begin{align}
		E_x &= E_r\cos\phi - E_\phi\sin\phi, \\
		E_y &= E_r\sin\phi + E_\phi\cos\phi.
	\end{align}

\bibliographystyle{agu08}
\bibliography{references}
\end{document}
