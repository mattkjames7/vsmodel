\documentclass[]{article}
\usepackage{natbib}
\usepackage{amsmath}
\usepackage{amsfonts}
\usepackage{amssymb}
%opening
\title{\texttt{vsmodel} Derivation}
\author{Matt James}

\begin{document}

\maketitle

This document describes the derivation of the model equations for the \texttt{vsmodel} package.

\section{Derivation of the Model}

In this section we derive the components of the Volland-Stern electric field model \citep{Volland1973,Stern1975} in polar coordinates (section \ref{SectPol}) and cylindrical coordinates (section \ref{SectCyl}).

As in \citet{Zhao2017}, we start with an electrostatic potential:

\begin{equation}
U(r,\phi) = -\frac{a}{r} - br^\gamma\sin{\phi},
\end{equation}

where $r$ is the radial distance in units of $R_E$; $\phi$ is the azimuth; $a$ is the corotation constant, $a$=92.4~kV~$R_E$; $\gamma$ is the shielding exponent ($\gamma=2$); and $b$ is related to the convection electric field using

\begin{equation}
	b = \frac{0.045}{(1.0 - 0.159 Kp + 0.0093 K_p^2)^3} \text{ (kV $R_E^{-2}$)}
\end{equation}

from \citet{Maynard1975}.

The electric field is then obtained using $\mathbf{E} = - \mathbf{\nabla} U$.

The following sections derive the radial, $E_r$, and azimuthal, $E_\phi$, components of the electric field using two slightly different methods, ultimately with the same result (thankfully) - this is likely due to the fact that we are only defining the two component in the equatorial plane.

\subsection{Polar Coordinates}
	\label{SectPol}
	In the case where we use polar coordinates, the electric field is calculated using:
	
	\begin{equation}
		\mathbf{E} =  - \mathbf{\nabla} U = -\frac{\partial U}{\partial r} \mathbf{\hat{r}} - \frac{1}{r} \frac{\partial U}{\partial \theta} \mathbf{\hat{\theta}} - \frac{1}{r \sin{\theta}} \frac{\partial U}{\partial \phi} \mathbf{\hat{\phi}}, \label{EqDelUP}
	\end{equation}
	
	where 
	
	\begin{align}
		\frac{\partial U}{\partial r} &= a r^{-2} - b \gamma r^{\gamma -1} \sin{\phi}, \\
		\frac{\partial U}{\partial \theta} &= 0, \\
		\frac{\partial U}{\partial \phi} & = -b r^{\gamma} \cos{\phi},
	\end{align}
	
	and because we are looking at the equatorial plane, $\theta = \pi$.
	
	Which means that the electric field components become
	
	\begin{align}
		E_r &= -a r^{-2} + b \gamma r^{\gamma-1} \sin{\phi}, \\
		E_\theta &= 0, \\
		E_\phi &= b r^{\gamma-1}\cos{\phi}.
	\end{align}
	
\subsection{Cylindrical Coordinates}
	\label{SectCyl}	
	Alternatively, we can try the same derivation using the cylindrical coordinate system to achieve effectively the same result. \textit{In all honestly - I'm not 100\% sure which I should be using, but both give the same result.}
	
	The cylindrical equivalent to equation \ref{EqDelUp}:
	\begin{equation}
		\mathbf{E} =  - \mathbf{\nabla} U = -\frac{\partial U}{\partial r} \mathbf{\hat{r}} - \frac{1}{r} \frac{\partial U}{\partial \phi} \mathbf{\hat{\phi}} - \frac{\partial U}{\partial z}\mathbf{\hat{z}}, \label{EqDelUC}
	\end{equation}
		
	where 
	
	\begin{align}
		\frac{\partial U}{\partial r} &= a r^{-2} - b \gamma r^{\gamma -1} \sin{\phi}, \label{EqdErC} \\
		\frac{\partial U}{\partial \phi} & = -b r^{\gamma} \cos{\phi}, \label{EqdEpC} \\
		\frac{\partial U}{\partial z} &= 0.  \label{EqdEzC}
	\end{align}
		
	From equations \ref{EqDelUC}, \ref{EqdErC}, \ref{EqdEpC} and \ref{EqdEzC}:
	
	
	\begin{align}
		E_r &= -a r^{-2} + b \gamma r^{\gamma-1} \sin{\phi}, \\
		E_\phi &= b r^{\gamma-1}\cos{\phi}, \\
		E_z &= 0.
	\end{align}

\section{SM Model Field}

	The previous section describes the model in polar/cylindrical coordinates (they are equivalent at $z=0$), here we convert the model to SM coordinates by rotating about the $z$-axis.
	
	Considering an Electric field vector $\mathbf{E}(r,\phi)$ with components $E_r$ and $E_\phi$ -- we need to rotate this vector by $\phi$ to transform into SM coordinates.
	
	Start by expressing the components of $\mathbf{E}$ in terms of some polar coordinates $\rho$ and $\alpha$:
	
	\begin{align}
		E_r &= \rho \cos{\alpha}, \label{EqEr}\\
		E_\phi &= \rho \sin{\alpha}, \label{EqEp}
	\end{align}
	
	then rotate by $\phi$,
	
	\begin{align}
		E_x &= \rho \cos{(\alpha + \phi)}, \label{EqEx0}\\
		E_y &= \rho \sin{(\alpha + \phi)}. \label{EqEy0}		
	\end{align}
	
	Using the trigonometric identities,
	
	\begin{align}
		\sin{(\alpha \pm \phi)} &= \sin\alpha\cos\phi \pm \cos\alpha\sin\phi, \\
		\cos{(\alpha \pm \phi)} &= \cos\alpha\cos\phi \mp \sin\alpha\sin\phi,
	\end{align}
	
	equations \ref{EqEx0} and \ref{EqEy0} become 
	
	\begin{align}
		E_x &= \rho\cos\alpha\cos\phi - \rho\sin\alpha\sin\phi, \\
		E_y &= \rho\sin\alpha\cos\phi + \rho\cos\alpha\sin\phi.
	\end{align}
	
	Then, substituting in equations \ref{EqEr} and \ref{EqEp}, gives
	
	\begin{align}
		E_x &= E_r\cos\phi - E_\phi\sin\phi, \\
		E_y &= E_r\sin\phi + E_\phi\cos\phi.
	\end{align}

\bibliographystyle{agu08}
\bibliography{references}
\end{document}
